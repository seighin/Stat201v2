\documentclass[aspectratio=169]{beamer}

\usepackage{shyne}

% Theme settings
\setbeamertemplate{navigation symbols}{}

\usetheme{Madrid}
\usefonttheme{structurebold}

\AtBeginSection[]
{ 	\begin{frame}{}

	{
	\usebeamerfont{frametitle}
	\begin{beamercolorbox}
		[wd={\textwidth}, center, sep=.2in, rounded=true, shadow=true]
		{frametitle}
	Week \thesection\\  \secname 
	\end{beamercolorbox}
	}
	
	\end{frame} 
}

\AtBeginSubsection[]
{ 	\begin{frame}{}

	{
	\usebeamerfont{frametitle}
	\begin{beamercolorbox}
		[wd={\textwidth}, center, sep=.2in, rounded=true, shadow=true]
		{frametitle}
	Section \thesection .\thesubsection\\  \subsecname 
	\end{beamercolorbox}
	}
	
	\end{frame} 
}

\title[Week 9]{Stat 201: Statistics I\\ Week 9 StatCrunch}
\author[M. Shyne]{}
\institute[Metro State]{\includegraphics[width=1.75in]{../images/metro_logo}}
\date[11/10/2019]{
\\ \bigskip \bigskip \includegraphics[width=.4in]{../images/cc_big}}
  

\begin{document}

\frame{\titlepage}

% Week 9
\setcounter{section}{8}
\section{Inference for Numerical Data}

%
% Section 9.1
%
\subsection{One sample hypothesis tests for means}


%%%%%%%%%%
\begin{frame}{Hypothesis tests for a mean in StatCrunch}

\begin{block}{}
\large
\begin{itemize}
\item Stat $\to$ T Stats $\to$ One Sample $\to$ With Summary\\ (or $\to$ With Data)
\item Enter ``Sample mean", ``Sample std. dev." and ``Sample size"\\
(or select column which contains data)
\item Select ``Hypothesis test for $\mu$"
\item Enter the appropriate values for null and alternative hypotheses.
\item Click ``Compute!"
\item The test statistic and p-value are found in ``T-Stat" and ``P-value"
\end{itemize}
\end{block}

\end{frame}

%
% Section 9.2
%
\subsection{Two sample hypothesis tests for means}

%%%%%%%%%%
\begin{frame}{Hypothesis tests for two means in StatCrunch}

\begin{block}{}
\large
\begin{itemize}
\item Stat $\to$ T Stats $\to$ Two Samples $\to$ With Summary
\item Enter ``Sample mean", ``Sample std. dev." and ``Sample size" for both samples
\item Leave ``Pool variances" unchecked
\item Select ``Hypothesis test for $\mu_1 - \mu_2$"
\item The null hypothesis should always be $H_0: \mu_1 - \mu_2 = 0$
\item Enter the appropriate value for the alternative hypothesis.
\item Click ``Compute!"
\item The test statistic and p-value are found in ``T-Stat" and ``P-value"
\end{itemize}
\end{block}

\end{frame}

%%%%%%%%%%
\begin{frame}{Confidence intervals for difference of means in StatCrunch}

\begin{block}{}
\large
\begin{itemize}
\item Stat $\to$ T Stats $\to$ Two Samples $\to$ With Summary
\item Enter ``Sample mean", ``Sample std. dev." and ``Sample size" for both samples
\item Leave ``Pool variances" unchecked
\item Select ``Confidence interval for $\mu_1 - \mu_2$"
\item Enter the appropriate confidence level. 
\begin{itemize}
\item Remember, for one-sided tests, the confidence level is (1 -2$\alpha$)\%.
\end{itemize}
\item Click ``Compute!"
\item The confidence interval bounds are found in ``L. Limit" and\\ ``U. Limit"
\end{itemize}
\end{block}

\end{frame}

%
% Section 9.3
%
\subsection{Hypothesis tests for paired samples}


%%%%%%%%%%
\begin{frame}{Hypothesis tests for matched pairs in StatCrunch}

\begin{block}{}
\large
\begin{itemize}
\item Stat $\to$ T Stats $\to$ Paired
\item Select columns of data for both samples
\item Select ``Hypothesis test for $\mu_D = \mu_1 - \mu_2$"
\item The null hypothesis should always be $H_0: \mu_D = 0$
\item Enter the appropriate value for the alternative hypothesis.
\item Click ``Compute!"
\item The test statistic and p-value are found in ``T-Stat" and ``P-value"
\end{itemize}
\end{block}

\end{frame}

%%%%%%%%%%
\begin{frame}{Confidence intervals for matched pairs in StatCrunch}
\begin{block}{}
\begin{itemize}
\large
\item Stat $\to$ T Stats $\to$ Paired
\item Select columns of data for both samples
\item Select ``Confidence interval for $\mu_D = \mu_1 - \mu_2$"
item Enter the appropriate confidence level.
\begin{itemize}
\item Remember, for one-sided tests, the confidence level is (1 -2$\alpha$)\%.
\end{itemize}
\item Click ``Compute!"
\item The confidence interval bounds are found in ``L. Limit" and\\ ``U. Limit"
\end{itemize}
\end{block}

\end{frame}


\end{document}