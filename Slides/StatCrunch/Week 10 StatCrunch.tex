\documentclass[aspectratio=169]{beamer}

\usepackage{shyne}

% Theme settings
\setbeamertemplate{navigation symbols}{}

\usetheme{Madrid}
\usefonttheme{structurebold}

\AtBeginSection[]
{ 	\begin{frame}{}

	{
	\usebeamerfont{frametitle}
	\begin{beamercolorbox}
		[wd={\textwidth}, center, sep=.2in, rounded=true, shadow=true]
		{frametitle}
	Week \thesection\\  \secname 
	\end{beamercolorbox}
	}
	
	\end{frame} 
}

\AtBeginSubsection[]
{ 	\begin{frame}{}

	{
	\usebeamerfont{frametitle}
	\begin{beamercolorbox}
		[wd={\textwidth}, center, sep=.2in, rounded=true, shadow=true]
		{frametitle}
	Section \thesection .\thesubsection\\  \subsecname 
	\end{beamercolorbox}
	}
	
	\end{frame} 
}

\title[Week 10]{Stat 201: Statistics I\\ Week 10 StatCrunch}
\author[M. Shyne]{}
\institute[Metro State]{\includegraphics[width=1.75in]{../images/metro_logo}}
\date[11/17/2019]{
\\ \bigskip \bigskip \includegraphics[width=.4in]{../images/cc_big}}
  

\begin{document}

\frame{\titlepage}

% Week 10
\setcounter{section}{9}
\section{Inference for Categorical Data}

%
% Section 10.1
%
\subsection{Hypothesis tests for proportions}


%%%%%%%%%%
\begin{frame}{Hypothesis tests for a proportion in StatCrunch}

\begin{block}{}
\large
\begin{itemize}
\item Stat $\to$ Proportion Stats $\to$ One Sample $\to$ With Summary
\item Enter ``\# of successes" and ``\# of observations"
\item Select ``Hypothesis test for p"
\item Enter the appropriate values for null and alternative hypotheses.
\item Click ``Compute!"
\item The test statistic and p-value are found in ``Z-Stat" and ``P-value"
\end{itemize}
\end{block}

\end{frame}

%%%%%%%%%%
\begin{frame}{Hypothesis tests for two proportions in StatCrunch}

\begin{block}{}
\begin{itemize}
\large
\item Stat $\to$ Proportion Stats $\to$ Two Samples $\to$ With Summary
\item Enter ``\# of successes" and ``\# of observations" for sample 1 and sample 2
\item Select ``Hypothesis test for $p_1 - p_2$"
\item The null hypothesis should always be $H_0: p_1 - p_2 = 0$
\item Enter the appropriate value for the alternative hypothesis.
\item Click ``Compute!"
\item The test statistic and p-value are found in ``Z-Stat" and ``P-value"
\end{itemize}
\end{block}

\end{frame}
%
% Section 10.2
%
\subsection{Goodness-of-Fit Tests}

%%%%%%%%%%
\begin{frame}{Goodness-of-fit tests in StatCrunch}

\begin{block}{}
\large
\begin{itemize}
\item Stat $\to$ Goodness-of-fit $\to$ Chi-Square Test
\item Select column that contains observed data
\item Specify expected distribution:
\begin{itemize}
\item For uniform distributions, select ``All cells in equal proportion"
\item For non-uniform distributions, select the column which contains expected frequencies
\end{itemize}
\item Leave default value of ``Expected" for display
\item Click ``Compute!"
\item The test statistic and p-value are found in ``Chi-Square" and ``P-value"
\end{itemize}
\end{block}
\end{frame}



%
% Section 10.3
%
\subsection{Tests for Independence}


%%%%%%%%%%
\begin{frame}{Tests for independence in StatCrunch}
\begin{block}{}
\large
\begin{itemize}
\item Stat $\to$ Tables $\to$ Contingency $\to$ With Summary
\item Select the columns that contain observed data
\item Select the column that contains row labels
\item If desired, select calculated values to be displayed\\(``Expected count" can be useful)
\item Leave ``Hypothesis tests" on default value of ``Chi-Square test for independence"
\item Click ``Compute!"
\item In the ``Chi-Square test" table, the test statistic and p-value are found in ``Value" and ``P-value"
\end{itemize}
\end{block}

\end{frame}


\end{document}