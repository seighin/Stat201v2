\documentclass[aspectratio=169]{beamer}

\usepackage{shyne}

% Theme settings
\setbeamertemplate{navigation symbols}{}

\usetheme{Madrid}
\usefonttheme{structurebold}

\AtBeginSection[]
{ 	\begin{frame}{}

	{
	\usebeamerfont{frametitle}
	\begin{beamercolorbox}
		[wd={\textwidth}, center, sep=.2in, rounded=true, shadow=true]
		{frametitle}
	Week \thesection\\  \secname 
	\end{beamercolorbox}
	}
	
	\end{frame} 
}

\AtBeginSubsection[]
{ 	\begin{frame}{}

	{
	\usebeamerfont{frametitle}
	\begin{beamercolorbox}
		[wd={\textwidth}, center, sep=.2in, rounded=true, shadow=true]
		{frametitle}
	Section \thesection .\thesubsection\\  \subsecname 
	\end{beamercolorbox}
	}
	
	\end{frame} 
}

\title[Week 1]{Stat 201: Statistics I\\ Week 1 StatCrunch}
\author[M. Shyne]{}
\institute[Metro State]{\includegraphics[width=1.75in]{../images/metro_logo}}
\date[8/25/2019]{
\\ \bigskip \bigskip \includegraphics[width=.4in]{../images/cc_big}}
  

\begin{document}

\frame{\titlepage}

%
% Week 1
%
\section{Introduction to MyStatLab and StatCrunch}

%
% Section 1.1
%
\subsection{Accessing MyStatLab}

%%%%%%%%%%
\begin{frame}{Accessing MyStatLab}
\begin{block}{}
\begin{itemize}
\item From D2L: Select ``MyStatLab" folder from the Content Browser
\item Click on the ``Pearson MyLab and Mastering" link
\item Click on the orange ``Open MyLab and Mastering" button
\item If this is the first time accessing MyStatLab, you will be asked to enter information necessary to gain access. Follow the instruction from the "Student Registration Handout".
\end{itemize}
\end{block}

\end{frame}

%%%%%%%%%%
\begin{frame}{Using MyStatLab}
\begin{block}{}
You can navigate to different sections of MyStatLab using the menu on the left side.
\begin{itemize}
\item Homework, quizzes and exams will be found in ``Assignments"
\item You can track your progress in ``Gradebook"
\item The online version of the textbook is found in ``eText"
\item The ``Study Plan" and ``Chapter Contents" sections contain additional material you might find useful. Nothing from these sections is required.
\end{itemize}
\end{block}

\end{frame}



% Section 2.3
\subsection{Accessing and loading data into StatCrunch}


%%%%%%%%%%
\begin{frame}{Accessing StatCrunch}
\begin{block}{}
Most of the time you will access StatCrunch directly from your homework (or quiz or exam). However, for in-class work or if you want to use it on your own, StatCrunch can be accessed from MyStatLab.
\begin{itemize}
\item Selecting ``StatCrunch" from the left-hand menu
\item Click the ``StatCrunch website" link 
\item Click ``Open StatCrunch" from the menu along the top of the page
\end{itemize}
StatCrunch presents a spreadsheet like page where data can be entered or loaded. The various statistical functions can be found from the menu along the top.
\end{block}

\end{frame}

%%%%%%%%%%
\begin{frame}{Loading data into StatCrunch}
\begin{block}{}
When working on assignments, data will be loaded automatically into StatCrunch. However, for in-class work, we will need to load data sets from D2L into StatCrunch ourselves.
\begin{itemize}
\item From D2L, select the ``Data" folder in the ``Content Browser".
\item Find the data file to be used. Either, click the down arrow next to the file name and select download, \bt{or} click on the file name and then click on either download button.
\item Do \bt{not} try to open the file from either browser or the download folder on your computer. This will open the file in Excel, which is not what we want.
\item From StatCrunch: Select Data $>$ Load $>$ From file $>$ on my computer
\item Click ``Choose file" and select the file from your hard drive (it will probably be in the downloads folder). 
\item Scroll to the bottom of the load data page (everything else on the page can be ignored). Click ``Load File"
\item A new StatCrunch tab will open with the data loaded.
\end{itemize}
\end{block}

\end{frame}


\end{document}