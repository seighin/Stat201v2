\documentclass[aspectratio=169]{beamer}

\usepackage{shyne}

% Theme settings
\setbeamertemplate{navigation symbols}{}

\usetheme{Madrid}
\usefonttheme{structurebold}

\AtBeginSection[]
{ 	\begin{frame}{}

	{
	\usebeamerfont{frametitle}
	\begin{beamercolorbox}
		[wd={\textwidth}, center, sep=.2in, rounded=true, shadow=true]
		{frametitle}
	Week \thesection\\  \secname 
	\end{beamercolorbox}
	}
	
	\end{frame} 
}

\AtBeginSubsection[]
{ 	\begin{frame}{}

	{
	\usebeamerfont{frametitle}
	\begin{beamercolorbox}
		[wd={\textwidth}, center, sep=.2in, rounded=true, shadow=true]
		{frametitle}
	Section \thesection .\thesubsection\\  \subsecname 
	\end{beamercolorbox}
	}
	
	\end{frame} 
}

\title[Week 7]{Stat 201: Statistics I\\ Week 7 StatCrunch}
\author[M. Shyne]{}
\institute[Metro State]{\includegraphics[width=1.75in]{../images/metro_logo}}
\date[10/13/2019]{
\\ \bigskip \bigskip \includegraphics[width=.4in]{../images/cc_big}}
  

\begin{document}

\frame{\titlepage}

% Week 7
\setcounter{section}{6}
\section{Estimating Population Parameters}

%
% Section 7.2
%
\setcounter{subsection}{1}
\subsection{Confidence Intervals}


%%%%%%%%%%
\begin{frame}{Confidence intervals of proportions in StatCrunch}
\begin{block}{}
\begin{itemize}
\item Stat $\to$ Proportion Stats $\to$ One Sample $\to$ With Summary
\item Enter ``\# of successes" and ``\# of observations"
\item Select ``Confidence interval for p"
\item Enter confidence level if different than 0.95.
\item Click ``Compute!"
\item The confidence interval is found in ``L. Limit" and ``U. Limit"
\end{itemize}
\end{block}
\end{frame}

%%%%%%%%%%
\begin{frame}{Confidence intervals of means in StatCrunch}
\begin{block}{}
\large
\begin{itemize}
\item Stat $\to$ T Stats $\to$ One Sample $\to$ With Summary \bt{or} With Data
\item For summary statistics (With Summary): Enter ``Sample mean", ``Sample std. dev." and ``Sample Size" (not degrees of freedom)
\item For data set (With Data): Select column which contains data
\item Select ``Confidence interval for $\mu$"
\item Enter confidence level if different than 0.95.
\item Click ``Compute!"
\item The confidence interval is found in ``L. Limit" and ``U. Limit"
\end{itemize}
\end{block}
\end{frame}



\end{document}