\documentclass[xcolor=table]{beamer}

\usepackage{shyne}

% Theme settings
\setbeamertemplate{navigation symbols}{}

\usetheme{Madrid}
\usefonttheme{structurebold}
\usefonttheme[onlymath]{serif}

\AtBeginSection[]
{ 	\begin{frame}{}

	{
	\usebeamerfont{frametitle}
	\begin{beamercolorbox}
		[wd={\textwidth}, center, sep=.2in, rounded=true, shadow=true]
		{frametitle}
	Chapter \thesection\\  \secname 
	\end{beamercolorbox}
	}
	
	\end{frame} 
}

\AtBeginSubsection[]
{ 	\begin{frame}{}

	{
	\usebeamerfont{frametitle}
	\begin{beamercolorbox}
		[wd={\textwidth}, center, sep=.2in, rounded=true, shadow=true]
		{frametitle}
	Section \thesection .\thesubsection\\  \subsecname 
	\end{beamercolorbox}
	}
	
	\end{frame} 
}

\title[Chapter 3]{Stat 201: Statistics I\\ StatCrunch: Chapter 3 }
\author[M. Shyne]{}
\institute[Metro State]{\includegraphics[width=1.75in]{../images/metro_logo}}
\date[8/5/2018]{
\\ \bigskip \bigskip \includegraphics[width=.4in]{../images/cc_big}}



\begin{document}
\frame{\titlepage}

% Chapter 3
\setcounter{section}{2}
\section{Statistics for Describing, Exploring, and Combining Data}

% Section 3.1
\subsection{Measures of Center}


\begin{frame}{Measures of center in StatCrunch}
\begin{block}{}
\begin{itemize}
\item Stat $\to$ Summary Stats $\to$ Columns
\item Select the column (or columns) which contains the data
\item Under ``Statistics" select statistics to calculate (ctrl-click to select multiple statistics)
\item For measures of center select ``Mean:, ``Median", ``Min", ``Max" and ``Mode"
\item Click ``Compute!"
\item Mean, median and mode will be displayed
\item Midrange must be calculated by hand using min and max
\end{itemize}
\end{block}

\begin{alertblock}{Note}
Mode will display ``No mode", ``Multiple modes" or the mode if exactly one exists. If there are multiple modes, you must identify them yourself.
\end{alertblock}
\end{frame}



% Section 3.2
\subsection{Measures of Variation}


\begin{frame}{Measures of variation in StatCrunch}
\begin{block}{}
\begin{itemize}
\item Stat $\to$ Summary Stats $\to$ Columns
\item Select the column (or columns) which contains the data
\item Under ``Statistics" select statistics to calculate (ctrl-click to select multiple statistics)
\item For measures of variation select ``Variance", ``Std. Dev." and ``Range"
\item Click ``Compute!"
\item The statistics will be displayed
\end{itemize}
\end{block}

\end{frame}

% Section 3.3
\subsection{Measures of Relative Standing and Boxplots}

\begin{frame}{Percentiles in StatCrunch}
\begin{block}{}
To find percentile of value $x$:
\begin{itemize}
\item If data is not ordered, order it using Data $\to$ Sort
\item The row number of the value \bt{before} $x$ is number of values $< x$
\item The row number of the last value is $n$
\item Use the formula to calculate percentile (round up)
\end{itemize}
To find value for percentile $p$:
\begin{itemize}
\item Stat $\to$ Summary Stats $\to$ Columns
\item Select the column (or columns) which contains the data
\item Under ``Percentiles..." enter $p$ (can enter multiple $p$'s separated by commas)
\item Click ``Compute!"
\item The value(s) will be displayed as ``$p$th Per."
\end{itemize}
\end{block}
\end{frame}


\begin{frame}{Five number summaries in StatCrunch}
\begin{block}{}
\begin{itemize}
\item Stat $\to$ Summary Stats $\to$ Columns
\item Select the column (or columns) which contains the data
\item Under ``Statistics" select ``Min", ``Q1", ``Median", ``Q3" and ``Max"
\item Or, under ``Percentile" enter ``1, 25, 50, 75, 99" (might not give correct values for the min and max in large data sets)
\item Click ``Compute!"
\item The values will be displayed, but maybe not in the right order
\end{itemize}
\end{block}

\end{frame}

\begin{frame}{Boxplots in StatCrunch}
\begin{block}{}
\begin{itemize}
\item Graph $\to$ Boxplot
\item Select the column (or columns) which contains the data
\item (Optional) Under ``Other options", click ``Draw boxes horizontally"
\item Click ``Compute!"
\item Hold pointer over plot to get IQR (inter-quartile range) and five number summary
\end{itemize}
\end{block}

\end{frame}


\end{document}