\documentclass[aspectratio=169]{beamer}

\usepackage{shyne}

% Theme settings
\setbeamertemplate{navigation symbols}{}

\usetheme{Madrid}
\usefonttheme{structurebold}

\AtBeginSection[]
{ 	\begin{frame}{}

	{
	\usebeamerfont{frametitle}
	\begin{beamercolorbox}
		[wd={\textwidth}, center, sep=.2in, rounded=true, shadow=true]
		{frametitle}
	Week \thesection\\  \secname 
	\end{beamercolorbox}
	}
	
	\end{frame} 
}

\AtBeginSubsection[]
{ 	\begin{frame}{}

	{
	\usebeamerfont{frametitle}
	\begin{beamercolorbox}
		[wd={\textwidth}, center, sep=.2in, rounded=true, shadow=true]
		{frametitle}
	Section \thesection .\thesubsection\\  \subsecname 
	\end{beamercolorbox}
	}
	
	\end{frame} 
}

\title[Week 4]{Stat 201: Statistics I\\ Week 4 StatCrunch}
\author[M. Shyne]{}
\institute[Metro State]{\includegraphics[width=1.75in]{../images/metro_logo}}
\date[9/29/2019]{
\\ \bigskip \bigskip \includegraphics[width=.4in]{../images/cc_big}}
  

\begin{document}

\frame{\titlepage}

%
% Week 4
%
\setcounter{section}{3}
\section{Examining and Summarizing Data}

%
% Section 4.1
%
\subsection{Summarizing and Plotting Data Distributions}

%%%%%%%%%%
\begin{frame}{Histograms in StatCrunch}
\begin{block}{}
\begin{itemize}
\item Graph $\to$ Histogram
\item Select column that contains data for histogram
\item Optional: Select type of histogram. This will adjust y-axis scale.
\item Optional: Set ``Bin: Width". This will determine number of bars displayed.
\item Click ``Compute!"
\end{itemize}
\end{block}

\begin{alertblock}{Note}
StatCrunch expects raw data for generating histograms. It won't work with data in frequency tables. To approximate a histogram using a frequency table, use a bar graph (see next section).
\end{alertblock}
\end{frame}


% 
% Section 4.2
%
\subsection{Summary Statistics}

%%%%%%%%%%
\begin{frame}{Measures of center in StatCrunch}
\begin{block}{}
\begin{itemize}
\item Stat $\to$ Summary Stats $\to$ Columns
\item Select the column (or columns) which contains the data
\item Under ``Statistics" select statistics to calculate (ctrl-click to select multiple statistics)
\item For measures of center select ``Mean:, ``Median", ``Min", ``Max" and ``Mode"
\item Click ``Compute!"
\item Mean, median and mode will be displayed
\item Midrange must be calculated by hand using min and max
\end{itemize}
\end{block}

\begin{alertblock}{Note}
Mode will display ``No mode", ``Multiple modes" or the mode if exactly one exists. If there are multiple modes, you must identify them yourself.
\end{alertblock}
\end{frame}

%%%%%%%%%%
\begin{frame}{Measures of variation in StatCrunch}
\begin{block}{}
\begin{itemize}
\item Stat $\to$ Summary Stats $\to$ Columns
\item Select the column (or columns) which contains the data
\item Under ``Statistics" select statistics to calculate (ctrl-click to select multiple statistics)
\item For measures of variation select ``Variance", ``Std. Dev." and ``Range"
\item Click ``Compute!"
\item The statistics will be displayed
\end{itemize}
\end{block}

\end{frame}

%
% Section 4.3
%
\subsection{Summarizing Data with Graphs}

%%%%%%%%%%
\begin{frame}{Categorical Graphs in StatCrunch}

\begin{block}{Bar plots and pie charts}
\begin{itemize}
\item Graph $\to$ Bar Plot or Pie Chart $\to$ With Summary
\item Select the column containing category names
\item Select the column containing category counts
\item Click ``Compute!"
\end{itemize}
\end{block}


\begin{block}{Pareto charts}
Follow steps for bar chart, except...
\begin{itemize}
\item Under ``Order by" select ``Count descending"
\end{itemize}
\end{block}
\end{frame}

%%%%%%%%%%
\begin{frame}{Paired Data Graphs in StatCrunch}
\begin{block}{Scatterplots and time series plots}
\begin{itemize}
\item Graph $\to$ Scatter plot
\item Select column that contains the data for the x-axis (time variable for time series plots)
\item Select column that contains the data for the y-axis
\item For time series plots, under ``Display" select lines (shift-click to select both points and lines)
\item Click ``Compute!"
\end{itemize}
\end{block}

\end{frame}



\end{document}