\documentclass[aspectratio=169]{beamer}

\usepackage{shyne}

% Theme settings
\setbeamertemplate{navigation symbols}{}

\usetheme{Madrid}
\usefonttheme{structurebold}

\AtBeginSection[]
{ 	\begin{frame}{}

	{
	\usebeamerfont{frametitle}
	\begin{beamercolorbox}
		[wd={\textwidth}, center, sep=.2in, rounded=true, shadow=true]
		{frametitle}
	Week \thesection\\  \secname 
	\end{beamercolorbox}
	}
	
	\end{frame} 
}

\AtBeginSubsection[]
{ 	\begin{frame}{}

	{
	\usebeamerfont{frametitle}
	\begin{beamercolorbox}
		[wd={\textwidth}, center, sep=.2in, rounded=true, shadow=true]
		{frametitle}
	Section \thesection .\thesubsection\\  \subsecname 
	\end{beamercolorbox}
	}
	
	\end{frame} 
}

\title[Week 5]{Stat 201: Statistics I\\ Week 5 StatCrunch}
\author[M. Shyne]{}
\institute[Metro State]{\includegraphics[width=1.75in]{../images/metro_logo}}
\date[10/6/2019]{
\\ \bigskip \bigskip \includegraphics[width=.4in]{../images/cc_big}}
  

\begin{document}

\frame{\titlepage}

%
% Week 5
%
\setcounter{section}{4}
\section{Relative Standing, Random Variables and Distributions}

%
% Section 5.1
%
\subsection{Measures of Relative Standing and Boxplots}

%%%%%%%%%%
\begin{frame}{Percentiles in StatCrunch}
\begin{block}{}
To find percentile of value $x$:
\begin{itemize}
\item If data is not ordered, order it using Data $\to$ Sort
\item The row number of the value \bt{before} $x$ is number of values $< x$
\item The row number of the last value is $n$
\item Use the formula to calculate percentile (round up)
\end{itemize}
To find value for percentile $p$:
\begin{itemize}
\item Stat $\to$ Summary Stats $\to$ Columns
\item Select the column (or columns) which contains the data
\item Under ``Percentiles..." enter $p$ (can enter multiple $p$'s separated by commas)
\item Click ``Compute!"
\item The value(s) will be displayed as ``$p$th Per."
\end{itemize}
\end{block}
\end{frame}


%%%%%%%%%%
\begin{frame}{Five number summaries in StatCrunch}
\begin{block}{}
\begin{itemize}
\item Stat $\to$ Summary Stats $\to$ Columns
\item Select the column (or columns) which contains the data
\item Under ``Statistics" select ``Min", ``Q1", ``Median", ``Q3" and ``Max"
\item Or, under ``Percentile" enter ``1, 25, 50, 75, 99" (might not give correct values for the min and max in large data sets)
\item Click ``Compute!"
\item The values will be displayed, but maybe not in the right order
\end{itemize}
\end{block}

\end{frame}

%%%%%%%%%%
\begin{frame}{Boxplots in StatCrunch}
\begin{block}{}
\begin{itemize}
\item Graph $\to$ Boxplot
\item Select the column (or columns) which contains the data
\item (Optional) Under ``Other options", click ``Draw boxes horizontally"
\item Click ``Compute!"
\item Hold pointer over plot to get IQR (inter-quartile range) and five number summary
\end{itemize}
\end{block}

\end{frame}


% 
% Section 4=5.2
%
\subsection{Probability Distributions}


%%%%%%%%%%
\begin{frame}{Mean and var. of random variables in StatCrunch}
\begin{block}{}
\begin{itemize}
\item Stat $\to$ Calculators $\to$ Custom
\item Under ``Values in:" select the column which contains the random variable values
\item Under ``Weights in:" select the column which contains the random variable probabilities
\item Click ``Compute!"
\item Mean and standard deviation will be displayed
\item Variance can be calculated by squaring the standard deviation
\end{itemize}
\end{block}
\end{frame}


\end{document}