\documentclass[xcolor=table]{beamer}

\usepackage{shyne}

% Theme settings
\setbeamertemplate{navigation symbols}{}

\usetheme{Madrid}
\usefonttheme{structurebold}
\usefonttheme[onlymath]{serif}

\AtBeginSection[]
{ 	\begin{frame}{}

	{
	\usebeamerfont{frametitle}
	\begin{beamercolorbox}
		[wd={\textwidth}, center, sep=.2in, rounded=true, shadow=true]
		{frametitle}
	Chapter \thesection\\  \secname 
	\end{beamercolorbox}
	}
	
	\end{frame} 
}

\AtBeginSubsection[]
{ 	\begin{frame}{}

	{
	\usebeamerfont{frametitle}
	\begin{beamercolorbox}
		[wd={\textwidth}, center, sep=.2in, rounded=true, shadow=true]
		{frametitle}
	Section \thesection .\thesubsection\\  \subsecname 
	\end{beamercolorbox}
	}
	
	\end{frame} 
}

\title[Chapter 7]{Stat 201: Statistics I\\ StatCrunch: Chapter 7 }
\author[M. Shyne]{}
\institute[Metro State]{\includegraphics[width=1.75in]{../images/metro_logo}}
\date[6/30/2019]{
\\ \bigskip \bigskip \includegraphics[width=.4in]{../images/cc_big}}


\begin{document}
\frame{\titlepage}

% Chapter 7
\setcounter{section}{6}
\section{Estimating Parameters and Determining Samples Sizes}


% Section 7.1
\subsection{Estimating a Population Proportion}


\begin{frame}{Confidence intervals of proportions in StatCrunch}
\begin{block}{}
\begin{itemize}
\item Stat $\to$ Proportion Stats $\to$ One Sample $\to$ With Summary
\item Enter ``\# of successes" and ``\# of observations"
\item Select ``Confidence interval for p"
\item Enter confidence level if different than 0.95.
\item Click ``Compute!"
\item The confidence interval is found in ``L. Limit" and ``U. Limit"
\end{itemize}
\end{block}
\end{frame}


\begin{frame}{Find needed sample size with StatCrunch}
\begin{block}{}
\large
\begin{itemize}
\item Stat $\to$ Proportion Stats $\to$ One Sample $\to$ Width/Sample Size
\item Enter ``Confidence level" if different than 0.95.
\item Enter estimated $\hat p$ as ``Target Proportion" 
\item Enter twice desired margin of error as ``Width"
\item Click ``Compute!"
\item The needed sample size is found in ``Sample size"
\end{itemize}
\end{block}

\end{frame}



% Section 7.2
\subsection{Estimating a Population Mean}


\begin{frame}{Confidence intervals of means in StatCrunch}
\begin{block}{}
\large
\begin{itemize}
\item Stat $\to$ Z Stats \bt{or} T Stats $\to$ One Sample $\to$ With Summary \bt{or}\\T Stats $\to$ One Sample $\to$ With Data
\item For known population standard deviation (Z Stats): Enter ``Sample mean", ``Standard deviation" and ``Sample Size" 
\item For unknown population standard deviation (T Stats): Enter ``Sample mean", ``Sample std. dev." and ``Sample Size" (not degrees of freedom)
\item For data set (T Stat with Data): Select column which contains data
\item Select ``Confidence interval for $\mu$"
\item Enter confidence level if different than 0.95.
\item Click ``Compute!"
\item The confidence interval is found in ``L. Limit" and ``U. Limit"
\end{itemize}
\end{block}
\end{frame}


\begin{frame}{Find needed sample size with StatCrunch}
\begin{block}{}
\begin{itemize}
\item Stat $\to$ Z Stats $\to$ One Sample $\to$ Width/Sample Size
\item Enter ``Confidence level" if different than 0.95.
\item Enter estimated or given standard deviation as ``Std. dev."
\item Enter twice desired margin of error as ``Width" 
\item Click ``Compute!"
\item The needed sample size is found in ``Sample size"
\end{itemize}
\end{block}

\end{frame}



\end{document}