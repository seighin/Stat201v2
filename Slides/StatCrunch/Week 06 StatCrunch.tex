\documentclass[aspectratio=169]{beamer}

\usepackage{shyne}

% Theme settings
\setbeamertemplate{navigation symbols}{}

\usetheme{Madrid}
\usefonttheme{structurebold}

\AtBeginSection[]
{ 	\begin{frame}{}

	{
	\usebeamerfont{frametitle}
	\begin{beamercolorbox}
		[wd={\textwidth}, center, sep=.2in, rounded=true, shadow=true]
		{frametitle}
	Week \thesection\\  \secname 
	\end{beamercolorbox}
	}
	
	\end{frame} 
}

\AtBeginSubsection[]
{ 	\begin{frame}{}

	{
	\usebeamerfont{frametitle}
	\begin{beamercolorbox}
		[wd={\textwidth}, center, sep=.2in, rounded=true, shadow=true]
		{frametitle}
	Section \thesection .\thesubsection\\  \subsecname 
	\end{beamercolorbox}
	}
	
	\end{frame} 
}

\title[Week 6]{Stat 201: Statistics I\\ Week 6 StatCrunch}
\author[M. Shyne]{}
\institute[Metro State]{\includegraphics[width=1.75in]{../images/metro_logo}}
\date[10/6/2019]{
\\ \bigskip \bigskip \includegraphics[width=.4in]{../images/cc_big}}
  

\begin{document}

\frame{\titlepage}

%
% Week 6
%
\setcounter{section}{5}
\section{Binomial and Normal Distributions}

%
% Section 6.1
%
\subsection{Binomial Probability Distributions}

%%%%%%%%%%
\begin{frame}{Binomial probabilities in StatCrunch}
\begin{block}{}
\begin{itemize}
\item Stat $\to$ Calculators $\to$ Binomial
\item Enter sample size (``n") and probability (``p")
\item Select appropriate comparison symbol and numeric value
\item Click ``Compute" if necessary
\item Probability will be displayed
\end{itemize}
\end{block}
\end{frame}


% 
% Section 6.2
%
\subsection{Normal Distributions}


%%%%%%%%%%
\begin{frame}{Probabilities of normal variables in StatCrunch}
\begin{block}{}
\begin{itemize}
\item Stat $\to$ Calculators $\to$ Normal
\item Default mean and standard deviation correspond to standard normal
\item For nonstandard distributions, enter mean and standard deviation
\item To find probabilities of ranges, select ``Between"
\item To find a probability greater or less than a z-score, choose appropriate comparison symbol and enter z-score inside parentheses (i.e. ``$P(X \le 1) =\ldots$")
\item To find a z-score corresponding to a probability, choose appropriate comparison symbol and enter probability \emph{outside} parentheses (i.e. ``$P(X \le \ldots) = 0.385$")
\item Click ``Compute"
\end{itemize}
\end{block}

\end{frame}



\end{document}