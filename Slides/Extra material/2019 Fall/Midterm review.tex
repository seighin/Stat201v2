\documentclass[xcolor=table, aspectratio=169, bigger]{beamer}

\usepackage{shyne}

% Theme settings
\setbeamertemplate{navigation symbols}{}

\usetheme{Madrid}
\usefonttheme{structurebold}
\usefonttheme[onlymath]{serif}

\AtBeginSection[]
{ 	\begin{frame}{}

	{
	\usebeamerfont{frametitle}
	\begin{beamercolorbox}
		[wd={\textwidth}, center, sep=.2in, rounded=true, shadow=true]
		{frametitle}
	Chapter \thesection\\  \secname 
	\end{beamercolorbox}
	}
	
	\end{frame} 
}

\AtBeginSubsection[]
{ 	\begin{frame}{}

	{
	\usebeamerfont{frametitle}
	\begin{beamercolorbox}
		[wd={\textwidth}, center, sep=.2in, rounded=true, shadow=true]
		{frametitle}
	Section \thesection .\thesubsection\\  \subsecname 
	\end{beamercolorbox}
	}
	
	\end{frame} 
}

\title[Midterm Review]{Stat 201: Statistics I\\Midterm Review}
\author[M. Shyne]{}
\institute[Metro State]{\includegraphics[width=1.75in]{../../images/metro_logo}}
\date[10/20/2019]{
\\ \bigskip \bigskip \includegraphics[width=.4in]{../../images/cc_big}}


\begin{document}
\frame{\titlepage}

%%%%%%%%%%
\begin{frame}{About the midterm exam}
\begin{block}{}
\begin{itemize}
\item Available on MyStatLab following class on 10/21
\item Due by end of day on 10/28
\item 26 questions, 100 points, covering weeks 1 - 6
\item Every question on exam has been a homework question, though the details will likely be different
\item Time limit: 4 hours, must be completed in one sitting
\item Can use any resource (book, notes, internet), except other people
\end{itemize}
\end{block}
\end{frame}

%%%%%%%%%%
\begin{frame}{Week 1}
\begin{block}{}
\begin{itemize}
\item Understand the potential for bias in studies
\item Be able to calculate and understand percentages
\item Know the difference between a parameter and a statistic
\item Identify type of variable
\begin{itemize}
\item Quantitative: Discrete or continuous
\item Categorical (Qualitative)
\end{itemize}
\item Identify levels of measurement
\begin{itemize}
\item Nominal
\item Ordinal
\item Interval
\item Ratio
\end{itemize}
\end{itemize}
\end{block}

\end{frame}

%%%%%%%%%%
\begin{frame}{Week 2}
\begin{block}{}
\begin{itemize}
\item Calculate probabilities:
\begin{itemize}
\item From a proportion (i.e. 3 in 12)
\item From a contingency table
\item Complements
\item Addition rule
\item Multiplication rule
\end{itemize}
\item Identify disjoint events
\item Identify independent and dependent events
\item Know how to identify and calculate probabilities of false positives and false negatives
\end{itemize}
\end{block}
\end{frame}

%%%%%%%%%%
\begin{frame}{Week 3}
\begin{block}{}
\begin{itemize}
\item Calculate conditional probabilities
\item Identify type of study
\begin{itemize}
\item Experimental
\item Observational
\end{itemize}
\end{itemize}
\end{block}
\end{frame}



%%%%%%%%%%
\begin{frame}{Week 4}
\begin{block}{}
\begin{itemize}
\item Create a frequency table
\item Create a histogram
\item From histogram, identify normal or skewed distributions and outliers
\item From a set of data, find (with proper units):
\begin{itemize}
\item Mean
\item Median
\item Mode
\item Midrange
\item Variance
\item Standard deviation
\item Range
\end{itemize}
\item Identify when to use mean/median
\item Identify proper graphs from data
\end{itemize}
\end{block}
\end{frame}


%%%%%%%%%%
\begin{frame}{Week 5}
\begin{block}{}
\begin{itemize}
\item Calculate 5 number summary and create corresponding boxplot
\item Find the mean and standard deviation for an arbitrary probability distribution
\item Find probabilities from an arbitrary probability distribution
\end{itemize}
\end{block}
\end{frame}


%%%%%%%%%%
\begin{frame}{Week 6}
\begin{block}{}
\begin{itemize}
\item Find probability of event from a binomial distribution
\item Find probability from standard normal, $z$, distribution
\item Find $z$-score which corresponds to given probability
\item Find probability of event from a non-standard normal distribution
\item Find value from non-standard normal distribution which corresponds to given probability
\end{itemize}
\end{block}
\end{frame}

\end{document}