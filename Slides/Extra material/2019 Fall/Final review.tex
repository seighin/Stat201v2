\documentclass[xcolor=table, aspectratio=169, bigger]{beamer}

\usepackage{shyne}

% Theme settings
\setbeamertemplate{navigation symbols}{}

\usetheme{Madrid}
\usefonttheme{structurebold}
\usefonttheme[onlymath]{serif}

\AtBeginSection[]
{ 	\begin{frame}{}

	{
	\usebeamerfont{frametitle}
	\begin{beamercolorbox}
		[wd={\textwidth}, center, sep=.2in, rounded=true, shadow=true]
		{frametitle}
	Chapter \thesection\\  \secname 
	\end{beamercolorbox}
	}
	
	\end{frame} 
}

\AtBeginSubsection[]
{ 	\begin{frame}{}

	{
	\usebeamerfont{frametitle}
	\begin{beamercolorbox}
		[wd={\textwidth}, center, sep=.2in, rounded=true, shadow=true]
		{frametitle}
	Section \thesection .\thesubsection\\  \subsecname 
	\end{beamercolorbox}
	}
	
	\end{frame} 
}

\title[Final Review]{Stat 201: Statistics I\\Final Review}
\author[M. Shyne]{}
\institute[Metro State]{\includegraphics[width=1.75in]{../../images/metro_logo}}
\date[11/30/2019]{}


\begin{document}
\frame{\titlepage}

%%%%%%%%%%
\begin{frame}{About the final exam}
\begin{block}{}
\begin{itemize}
\item Available on MyStatLab following class on 12/2
\item Due by end of day on 12/9
\item 23 questions, 100 points
\begin{itemize}
\item 5 questions, 18 points from pre midterm
\item 18 questions, 82 points from post midterm
\end{itemize}
\item Every question on exam has been a homework question, though the details will likely be different
\item Time limit: 4 hours, must be completed in one sitting
\item Can use any resource (book, notes, internet), except other people
\end{itemize}
\end{block}
\end{frame}


%%%%%%%%%%
\begin{frame}{Pre midterm}
\begin{block}{}
\begin{itemize}
\item Calculate probabilities:
\begin{itemize}
\item From a contingency table
\item Addition rule
\item Multiplication rule
\end{itemize}
\item Calculate summary statistics:
\begin{itemize}
\item Mean
\item Median
\item etc.
\end{itemize}
\item Find probability from normal distribution (critical values)
\item Find probability from binomial distribution
\end{itemize}
\end{block}
\end{frame}

%%%%%%%%%%
\begin{frame}{Week 7}
\begin{block}{}
\begin{itemize}
\item Identify parameters of sampling distributions
\item Calculate probabilities of sample statistics
\item Construct confidence intervals for means or proportions
\item Correctly interpret a confidence interval
\end{itemize}
\end{block}
\end{frame}

%%%%%%%%%%
\begin{frame}{Week 8}
\begin{block}{}
\begin{itemize}
\item Calculate sample sizes for given scenarios
\item Hypothesis testing:
\begin{itemize}
\item Identify the null and alternative hypotheses
\item Calculate a test statistic and p-value
\item Make a decision based on p-value and significance level
\item State conclusion in terms of research question
\end{itemize}
\item Understand type I and type II errors
\end{itemize}
\end{block}
\end{frame}

%%%%%%%%%%
\begin{frame}{Week 9}
\begin{block}{}
\begin{itemize}
\item Test a claim about a population mean from summary statistics or sample data
\item Test a claim about two population means using two independent samples
\item Test a claim about the difference between populations using samples of paired data
\end{itemize}
\end{block}
\end{frame}


%%%%%%%%%%
\begin{frame}{Week 10}
\begin{block}{}
\begin{itemize}
\item Test a claim about two population proportions
\item Test a claim about population proportion
\item Test the fit of a sample frequency distribution to an expected distribution
\item Test independence of two factors using a sample contingency table

\end{itemize}
\end{block}
\end{frame}

%%%%%%%%%%
\begin{frame}{Week 11}
\begin{block}{}
\begin{itemize}
\item Plot paired data in a scatterplot
\item Calculate correlation coefficient of a sample
\item Test whether population correlation parameter $\rho$ is zero or not
\item Calculate regression equation from sample
\item Make predictions for the response variable given a predictor value and regression results
\end{itemize}
\end{block}
\end{frame}




\end{document}