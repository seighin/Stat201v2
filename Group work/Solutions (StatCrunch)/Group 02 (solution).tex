\documentclass{article}

\usepackage{shyne}

% document format
\topmargin 0in
\oddsidemargin 0in
\evensidemargin 0in
\headheight 0in
\headsep 0in
\topskip 0in
\textheight 9in
\textwidth 6.5in
\linespread{1.3}

\begin{document}

\begin{flushleft}
\section*{Group Work - Week 2}
\paragraph{1} Consider rolling a fair six-sided die.
\begin{enumalpha}
\item Let event $A$ be rolling an even number. What is a trial for this scenario? What is the sample space? Is $A$ a simple event? What is $P(A)$? What is $\bar A$, the complement of $A$? What is $P(\bar A)$? Is $A$ unlikely? Is $A$ unusual?\\
\medskip
\bt{Trial: One roll of the die\\
Sample space: Roll is one of \{1, 2, 3, 4, 5, 6\} \\
$\bv A$ is not a simple event. It is composed of the simple events of rolling 2, 4 or 6.}\\
$\bv{ P(A) = \frac 3 6 = 0.5}$\\
\bt{$\bv{\bar A}$ is the event of rolling an odd number (1, 3, 5).}\\
$\bv{ P(\bar A) = \frac 3 6 = 0.5}$\\
\bt{$\bv A$ is neither unlikely nor unusual.}

\vspace{.5in}
\item Let event $A$ be rolling an even number. Let event $B$ be rolling a 3. Are events $A$ and $B$ disjoint? What is $P(A \text{ or } B)$?\\
\medskip
\bt{Events $\bv A$ and $\bv B$ are disjoint. A roll can't be even and 3 at the same time.}\\
\medskip $\ds \bv{P(A \text{\bt{ or }} B) = P(A) + P(B) =  \frac 3 6 + \frac 1 6 = \frac 4 6 = \frac 2 3 = 0.666\ldots}$
\vspace{.5in}
\item Consider rolling a die twice. Let event $A$ be getting an even number on the first roll. Let event $B$ be getting 5 or more on the second roll. Are events $A$ and $B$ independent? What is $P(A \text { and } B)$?\\
\medskip
\bt{Events $\bv A$ and $\bv B$ are independent. The outcome of the first roll has no effect on the outcome of the second roll.}\\
\medskip $\ds \bv{P(A \text{\bt{ and }} B) = P(A) \times P(B) = \frac 3 6 \times \frac 2 6 = \frac {6}{36} = \frac 1 6 = 0.166\ldots}$
\end{enumalpha}

\newpage
\paragraph{2} Consider a standard deck of playing cards... 52 cards, 4 suits of 13 cards each, 3 cards of each suit are face cards, 2 suits are black (clubs and spades) and 2 are red (hearts and diamonds).
\begin{enumalpha}
\item Let event $A$ be drawing a random card that is a diamond. What is a trial for this scenario? What is the sample space? Is $A$ a simple event? What is $P(A)$? What is $\bar A$, the complement of $A$? What is $P(\bar A)$? Is $A$ unlikely? Is $A$ unusual?\\
\medskip
\bt{Trial: The drawing of one card\\
Sample space: The card drawn is one of the 52 cards in the deck \\
$\bv A$ is not a simple event. It is composed of the simple events of drawing any one of the 13 diamonds.}\\
$\bv{ P(A) = \frac {13}{52} = 0.25}$\\
\bt{$\bv{\bar A}$ is the event of drawing one of the 39 cards that are not diamonds.}\\
$\bv{ P(\bar A) = \frac{39}{52}  = 0.75}$\\
\bt{$\bv A$ is neither unlikely nor unusual.}\\
\bigskip \bt{Or if one only considers suit and ignores values of cards\ldots}\\
\bigskip
\bt{Trial: The drawing of one card\\
Sample space: The card drawn is a club, spade, heart or diamond. \\
$\bv A$ is a simple event. It can not be simplified if one is only considering suits of cards.}\\
$\bv{ P(A) = \frac {1}{4} = 0.25}$\\
\bt{$\bv{\bar A}$ is the event of drawing a club, spade or heart.}\\
$\bv{ P(\bar A) = \frac{3}{4}  = 0.75}$\\
\bt{$\bv A$ is neither unlikely nor unusual.}\\

\vspace{.5in}
\item Let event $A$ be drawing a random card that is a diamond. Let event $B$ be drawing a random card that is a face card. Are events $A$ and $B$ disjoint? What is $P(A \text{ or } B)$?\\
\medskip
\bt{Events $\bv A$ and $\bv B$ are not disjoint. A card can be both a diamond and a face card.}\\
\medskip $\ds \bv{P(A \text{\bt{ or }} B) = P(A) + P(B) - P(A \btext{ and B)} =  \frac {13}{52} + \frac {12}{52} - \frac {3}{52} = \frac {22}{52} \approx 0.423}$
\vspace{.5in}
\newpage
\item Consider drawing three cards. Let event $A$ be the first card is a heart. Let event $B$ be the second card is a club. Let event $C$ be the third card is black. Are events $A$, $B$ and $C$ independent? What is $P(A \text { and } B \text { and } C)$?\\
\medskip
\bt{Events $\bv A$, $\bv B$ and $\bv C$ are not independent. The drawing of each card affects the probabilities of each subsequent draw.}\\
\bigskip
$\ds \bv{P(A) = \frac {13}{52} = \frac 1 4, \qquad P(B \mid A) = \frac {13}{51}, \qquad P(C \mid A \btext{ and } B) = \frac {25}{50} = \frac 1 2}$\\
\bigskip
$\ds \bv{P(A \btext{ and } B \btext{ and } C) = P(A) \times P(B \mid A) \times P(C \mid A \btext{ and } B)}$\\
\medskip
$\ds \bv{ \qquad \qquad \qquad \qquad \qquad= \frac 1 4 \times \frac {13}{51} \times \frac 1 2 = \frac {13}{408} \approx 0.0319}$

\end{enumalpha}

\newpage
\paragraph{3} The data set ``hair\_eye.csv" on D2L contains the hair and eye colors, as well as sex, of a sample of statistics students. Below is a table showing the distributions of students by eye color and gender.\\
\medskip
\renewcommand{\arraystretch}{1}
{\centering
\begin{tabular}{ r| c c c c | c}
& \multicolumn{4}{c}{Eye color}\\
Gender & Blue & Brown & Green & Hazel & Total\\
\hline
 Female &  114   &  122  &  31  &  46 & 313\\
 Male   & 101  &  98  &  33  &  47 & 277\\
 \hline
 Total	& 215 & 220 & 64 & 93 & 592 
\end{tabular}
\par}
\begin{enumalpha}
\item Let event $A$ be a randomly selected student having green eyes. What is a trial for this scenario? What is the sample space? Is $A$ a simple event? What is $P(A)$? What is $\bar A$, the complement of $A$? What is $P(\bar A)$? Is $A$ unlikely? Is $A$ unusual?\\
\medskip
\bt{Like with the previous playing card example, there are two ways of treating this problem. We'll only consider the simpler here.\\}
\medskip
\bt{Trial: Selecting one student.\\
Sample space: The student's eye color is blue, brown, green or hazel. \\
$\bv A$ is a simple event.}\\
$\bv{ P(A) = \frac {64}{592} \approx 0.108}$\\
\bt{$\bv{\bar A}$ is the event of drawing one of the 39 cards that are not diamonds.}\\
$\bv{ P(\bar A) = \frac{528}{592}  \approx 0.892}$\\
\bt{$\bv A$ is somewhat unlikely, but not below 5\%. $\bv A$ is not unusual.}\\
\vspace{.5in}

\item Let event $A$ be a randomly selecting a student with brown or blue eyes. Let event $B$ be a randomly selecting a female student. Are events $A$ and $B$ disjoint? What is $P(A \text{ or } B)$?\\
\medskip
\bt{Events $\bv A$ and $\bv B$ are not disjoint. A student can both have brown or blue eyes and be female.}\\
\medskip $\ds \bv{P(A \text{\bt{ or }} B) = P(A) + P(B) - P(A \btext{ and B)} }$\\
\medskip $\ds \bv{\qquad \qquad \qquad =  \frac {215 + 220}{592} + \frac {313}{592} - \frac {114+122}{592} = \frac {512}{592} \approx 0.865}$
\vspace{.5in}

\item Consider randomly selecting two students. Let event $A$ be the first student has blue eyes. Let event $B$ be the second student has hazel eyes. Are events $A$ and $B$ independent? What is $P(A \text { and } B)$?\\
\medskip
\bt{Events $\bv A$ and $\bv B$ are technically not independent, because the sample size (2) is less than 5\% of the population (592), we can treat them as independent. }\\
\bigskip
$\ds \bv{P(A \btext{ and } B) = P(A) \times P(B) = \frac {215}{592} \times \frac {93}{592} = \frac {19995}{350464} \approx 0.0571}$

\end{enumalpha}



\end{flushleft}
\end{document}