\documentclass{article}

\usepackage{shyne}

% document format
\topmargin 0in
\oddsidemargin 0in
\evensidemargin 0in
\headheight 0in
\headsep 0in
\topskip 0in
\textheight 9in
\textwidth 6.5in
\linespread{1.3}

\begin{document}

\begin{flushleft}
\section*{Group Work - Week 1}
\paragraph{1} The Tortilla and Cheese Organization (TACO) is a taco advocacy and lobbying group. They want to understand the relationship between taco consumption and grades. They begin by randomly selecting one stats class, one writing class, one business class and one nursing class at Metro State. They go to one class each week over four weeks (the stats class the first week, the writing class the second, etc) and ask each student attending the class their age, gender and "Knowing that tacos are delicious, how many tacos have you eaten in the past month?" At the end of the semester, they get the average grade for each class.\\
\medskip
After analyzing the data, TACO finds that the stats class has a class GPA 0.1 higher than the other classes and had consumed more tacos, on average, than the other classes. The results were statistically significant. They issue a press release declaring, "Eating tacos leads to higher grades."

\begin{itemize}
\item [(a)] Identify the research question, the population being studied, the data used to answer the question and any potential problems or pitfalls with the study.
\vspace{2.5in}
\item[(b)] For each variable used by the study, identify the type (quantitative/categorical, discrete/continuous, etc) and the level of measurement.
\end{itemize}


\newpage
\paragraph{2} A medical research company has developed what they hope is a promising new drug to treat a disease which has a life expectancy of 2 years after diagnosis. After conducting safety trials, they conduct a efficacy study. At four urban medical centers in different region of the US, they recruit a total of 176 patients who are newly diagnosed with the disease. The patients are randomly selected to either receive the new drug or to receive a placebo. All patients continue to receive the current standard of care. Neither the patients nor their doctors know which group they are in. Every patient in the study has a doctor visit every other month where vital measurements are recorded such as weight, blood pressure, blood sodium levels, as well as life style information such as marital status, education level and alcohol consumption. Also, if a patient dies during the study, the time between diagnosis and death is recorded.\\
\medskip
After a six year study, 39 of the patients dropped out of the study or lost contact with the researchers. It is found that patients in the treatment group lived an average of 13 months longer. A parallel NIH funded study conducted at a university is still ongoing.

\begin{itemize}
\item [(a)] Identify the research question, the population being studied, the data used to answer the question and any potential problems or pitfalls with the study.
\vspace{2.5in}
\item[(b)] For some of the variables used by the study, identify the type (quantitative/categorical, discrete/continuous, etc) and the level of measurement.
\end{itemize}


\newpage
\paragraph{3} Wanting to better understand the risk factors associated with various kinds of heart disease, researchers at the National Heart, Lung, and Blood Institute selected a small town in Massachusetts. Five thousand residents were selected for the study, an equal number of men and women and representative proportions for age groups. Every two years, study subjects undergo a detailed medical history, physical examination, and medical tests. Data collected include disease history for coronary heart disease, stroke and mental illness, vital stats such as weight, blood pressure and body temperature, lifestyle information such as history of smoking, diet and exercise, and age, gender and other biographical information.\\
\medskip
The study is ongoing, but some of the results from it include smoking is associated with an increase risk of heart disease and exercise is associated with a decreased risk of heart disease, high blood pressure increases risk of stroke, and mental illness affects heart health.

\begin{itemize}
\item [(a)] Identify the research question, the population being studied, the data used to answer the question and any potential problems or pitfalls with the study.
\vspace{2.5in}
\item[(b)] For some of the variables used by the study, identify the type (quantitative/categorical, discrete/continuous, etc) and the level of measurement.
\end{itemize}
\end{flushleft}
\end{document}